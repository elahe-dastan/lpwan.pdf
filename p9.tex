\قسمت{سوال نهم}

پروژه \متن‌لاتین{Connected Ships} در سال‌های ۲۰۱۹ تا ۲۰۲۱ در نروژ عملیاتی شده است.
کشتی‌های بزرگ سوخت زیادی مصرف می‌کنند و از همین رو تاثیر زیادی در گرم شدن کره زمین دارند.
یکی از هزینه‌های اصلی شرکت‌های کشتی‌رانی همین هزینه‌ی سوخت است و از این رو بهبود بهینه‌سازی مصرف
یکی از نوآوری‌های مهم در حوزه‌ی کشتی‌رانی است.

در این پرژه سنسورهایی روی کشتی متصل شده است که با جمع‌آوری داده‌های غیرحیاتی آن‌ها در اختیار مرکز در ساحل قرار می‌دهند.
برای این ارتباط از شبکه‌ی ماهواره‌ای استفاده شده است چرا که در اقیانوس هیچ پوشش شبکه‌ای دیگری وجود ندارد.
اما مساله دیگری ارتباط خود سنسورها در کشتی است که این برای جلوگیری از صرف هزینه‌ی زیاد و کابل کشی از تکنولوژی‌های \متن‌لاتین{LPWAN}
استفاده خواهد شد که زیرساخت آماده‌ای برای این ارتباط فراهم می‌کنند.

در نهایت با جمع‌اوری این داده نه تنها از یک کشتی بلکه از یک ناوگان درهای زیادی برای کار بر روی این داده‌ّها باز می‌شود و هم اکنون نیز در این
پروژه تعداد زیادی همکاری برای بحث‌های هوش مصنوعی و پردازش داده وجود دارد.
