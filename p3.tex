\قسمت{سوال سوم}

در شبکه‌های \متن‌لاتین{LoRaWAN} سه کلاس کاری می‌توان برای اشیا در نظر گرفت.

\شروع{فقرات}
\فقره در کلاس ($A$ یا \متن‌لاتین{All}) شی هر زمان که به خواهد شروع به ارسال داده کرده و دو پریود متوالی آینده را برای دریافت \متن‌لاتین{Downlink} خواهد داشت. این کلاس پایین‌ترین مصرف انرژی را دارد چرا که شی تنها در زمان‌هایی که لازم است
روشن می‌شود و می‌تواند دوباره خاموش شود. با توجه به ساختار دریافت \متن‌لاتین{Downlink} در این کلاس می‌توان به سادگی برای پیام‌ها \متن‌لاتین{Ack} دریافت کرد اما برای سایر پیام‌های \متن‌لاتین{Downlink} می‌بایست تا تصمیم شی برای ارسال داده صبر کرد.
\فقره در کلاس ($B$ یا \متن‌لاتین{Beacon}) نود به صورت همگام می‌تواند \متن‌لاتین{Downlink} دریافت کند برای اینکار نود به جز دو بازه دریافت که در کلاس $A$ تعریف شده بود یک بازه دریافت قابل پیش‌بینی نیز دارد.
این کلاس مصرف بالاتری دارد چرا که نیاز است یک بازه دوره‌ای برای دریافت \متن‌لاتین{Downlink} روشن شود. مزیت این کلاس قابلیت دریافت \متن‌لاتین{Downlink} حتی در زمان‌هایی که ارسالی ندارد می‌باشد.
\فقره در کلاس ($C$ یا \متن‌لاتین{Continues}) بیشترین مصرف توان را داشته و شی در هر زمان می‌تواند داده دریافت کند.
\پایان{فقرات}

