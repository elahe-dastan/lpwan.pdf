\قسمت{سوال دوم}

در بسته‌های \متن‌لاتین{LoRa} از تصحیح خطا جلورونده یا مختصرا \متن‌لاتین{FEC} استفاده می‌شود.
در این فرآیند بیت‌های تصحیح خطا به داده‌های ارسال اضافه می‌شوند.
این بیت‌های اضافه شده کمک می‌کنند تا داده‌های از دست رفته به خاطر تداخل بازگردانی شوند.
بیت‌های بیشتر این پروسه بازگردانی را ساده‌تر می‌کنند اما باعث هدر رفت پهنای باند و عمر باتری می‌شوند.
در \متن‌لاتین{LoRa} ما نرخ‌های کدگذاری $4/5$، $4/6$، $4/7$ و $4/8$ را داریم.

\begin{table}
\caption{توانایی \متن‌لاتین{LoRa} در تشخیص و تصحیح خطا \مرجع{Pham2020}}
\begin{latin}\begin{tabularx}
  {\textwidth}
  {|*{3}{X|}}
  \toprule
  Coding rates &
  Error detection (bits) &
  Error correction (bits) \\
  \midrule
  $4/5$ &
  0 &
  0 \\
  \midrule
  $4/6$ &
  1 &
  0 \\
  \midrule
  $4/7$ &
  2 &
  1 \\
  \midrule
  $4/8$ &
  3 &
  1 \\
  \bottomrule
\end{tabularx}\end{latin}
\end{table}

پهنانی باند در \متن‌لاتین{LoRa} می‌توان بین ۱۲۵ تا ۵۰۰ کیلوهرتز باشد و با توجه به استفاده از باند بدون لایسنس این پهنای باند وابسته به پارامتر‌های منطقه‌ای و فاکتور گسترش می‌باشد.

در متن‌لاتین{LoRa} نرخ باد یا نرخ علائم از رابطه‌ی زیر محاسبه می‌گردد:

\begin{align}
  \label{معادله: نرخ باد یا علائم در LoRa}
  R_{s} = BW / 2^{SF}
\end{align}

که در آن \متن‌لاتین{BW} پهنای باند و \متن‌لاتین{SF} فاکتور گسترش می‌باشد.
\مرجع{Augustin2016}

در ادامه نرخ داده‌ی ارسالی را می‌توان با استفاده از رابطه زیر محاسبه کرد:

\begin{align}
  \label{معادله: نرخ داده در LoRa}
  R_{b} = SF \times \frac{BW}{2^{SF}} \times CR
\end{align}

در این رابطه \متن‌لاتین{CR} نرخ کدگذاری، \متن‌لاتین{SF} فاکتور گسترش و \متن‌لاتین{BW} پهنای باند می‌باشد.
\مرجع{Augustin2016}

\شروع{شکل}
\درج‌تصویر[width=.5\textwidth]{./img/lora-packet.png}
\تنظیم‌ازوسط
\شرح{ساختار بسته \متن‌لاتین{LoRa} \مرجع{Augustin2016}}
\پایان{شکل}

رابطه زیر مشخص می‌کند برای ارسال یک داده به چه تعداد علامت نیاز داریم. این پارامتر با $n_{s}$ نمایش داده می‌شود.

\begin{align}
  \label{معادله: تعداد علائم مورد نیاز در LoRa}
  n_{s} = 8 + \max\left( \left\lceil \frac{8PL - 4SF + 8 + CRC + H}{4 \times (SF - DE)} \right\rceil \times \frac{4}{CR}, 0 \right)
\end{align}

در این رابطه در صورت فعال بودن \متن‌لاتین{CRC} مقدار آن برابر ۱۶ و در غیر این صورت برابر صفر است.
\متن‌لاتین{CR} نرخ کدگذاری،
\متن‌لاتین{PL} اندازه داده،
\متن‌لاتین{SF} فاکتور گسترش است.
در این رابطه \متن‌لاتین{H} اندازه سرآیند بوده که در صورت فعال بودن برابر ۲۰ و در غیر این صورت صفر است.
در این رابطه \متن‌لاتین{DE} در صورت فعال بودن حالت نرخ داده پایین یا \متن‌لاتین{low data rate} برابر ۲ و در غیر این صورت برابر صفر است.
\مرجع{Augustin2016}
\مرجع{Pham2020}

