\قسمت{سوال چهارم}

یکی از تکنیک‌ها در شبکه‌های بی‌سیم استفاده از \متن‌لاتین{Frequency Hopping} است. در این تکنیک با هماهنگی در میان ارسال کننده و گیرنده فرکانس‌های ارسال در زمان
تغییر می‌کند. پیاده‌سازی این شیوه در شبکه‌های \متن‌لاتین{LoRa} در قالب \متن‌لاتین{LR-FHSS} یا
\متن‌لاتین{Frequency Hopping Spread Spectrum}
صورت می‌پذیرد. در این روش هر کانال به تعدادی زیرکانال شکسته شده و سرآینده بسته روی همه این زیرکانال‌ها ارسال می‌شود.
خود داده اما قطعه قطعه شده و هر قطعه به وسیله‌ی یک زیرکانال ارسال می‌گردد.
از آنجایی که \متن‌لاتین{Gateway} روی همه‌ی این کانال‌ها گوش می‌دهد می‌تواند بسته را دوباره بازسازی کند.
